\documentclass[a4paper, 11pt]{article}
\usepackage{graphicx}
\title{Using DIALS}
\author{Graeme Winter, Diamond Light Source}

\begin{document}

\maketitle

\section{Caveat Lector}

These are my notes on how to run elements of the DIALS processing suite. I make no promises that anything described in here will work, or will give sensible results, or will in fact do anything. You have been warned.

\section{Introduction}

DIALS processing may be performed by either running the individual tools (spot finding, indexing, refinement, integration, exporting to MTZ) or you can run the whole lot through \verb|dials.process|, which just chains them together (and incidentally does all of the processing in P1.)

\section{dials.process}

In the simplest case, \verb|dials.process /here/are/all/images*.cbf| will do  sensible processing, with a static model of the experiment and sample, and will output a reflection file \verb|integrated.mtz| containing the intensity measurements assuming everything works correctly. Some sensible options to use are:

\begin{itemize}
\item{\verb|scan_varying=true| - allow the crystal orientation and unit cell constants to vary during the scan}
\item{\verb|--nproc=1| - only use one processor (necessary currently for data in NeXus files}
\item{\verb|intensity.algorithm=sum2d| - use 2D summation integtration rather than the default 3D summation, other algorithms are being added}
\item{\verb|shoebox.n_blocks=N| - for some N, split the data set into N blocks for integration, so as not to overload the computer}
\item{\verb|-i| - pass the images to process through the standard input e.g. from \verb|find . -name '*.cbf'| to avoid issues with limited command-line lengths}
\end{itemize}

\noindent
I have used all of these options and they appear to do sensible things. Important to make sure that what is passed in is exactly one sweep.

\section{Results}

At the end of processing you will see something like:

{\small
\begin{verbatim}
Space group symbol from file: P1
Space group number from file: 1
Space group from matrices: P 1 (No. 1)
Point group symbol from file: 1
Number of batches: 540
Number of crystals: 1
Number of Miller indices: 318767
Resolution range: 150.015 1.169
History:
Crystal 1:
  Name: XTAL
  Project: DIALS
  Id: 1
  Unit cell: (57.8091, 57.7715, 150.015, 89.9809, 89.9967, 90.01)
  Number of datasets: 1
  Dataset 1:
    Name: FROMDIALS
    Id: 1
    Wavelength: 0.97625
    Number of columns: 11
    label        #valid  %valid     min     max type
    H            318767 100.00%  -47.00   39.00 H: index h,k,l
    K            318767 100.00%  -34.00   43.00 H: index h,k,l
    L            318767 100.00% -114.00  114.00 H: index h,k,l
    M_ISYM       318767 100.00%    1.00    1.00 Y: M/ISYM, packed partial/reject flag and symmetry number
    BATCH        318767 100.00%    2.00  538.00 B: BATCH number
    I            318767 100.00%   -8.34 3032.36 J: intensity
    SIGI         318767 100.00%    0.00   55.21 Q: standard deviation
    FRACTIONCALC 318767 100.00%    1.00    1.00 R: real
    XDET         318767 100.00%    3.38 2458.38 R: real
    YDET         318767 100.00%    2.97 2522.77 R: real
    ROT          318767 100.00%   82.09  162.45 R: real
\end{verbatim}
}

\noindent
which just shows the summary of what is in the output MTZ file. There are also indexing and refinement results which I need to add FIXME.

\section{What to do Next}

The output MTZ file \verb|integrated.mtz| may be read by pointless wich will assign the correct symmetry and so on, and may then be scaled with aimless. I have been running:

{\small
\begin{verbatim}
pointless hklin integrated.mtz hklout sorted.mtz
aimless hklin sorted.mtz hklout scaled.mtz
\end{verbatim}
}

\noindent
to get merged data for downstream analysis. The output from this will include the merging statistics which will give some idea of the data quality. Often passing in a sensible resolution limit to aimless is also helpful...

\end{document}
