\documentclass[a4paper, 11pt]{article}
\usepackage{graphicx}
\title{Using DIALS}
\author{Graeme Winter, Diamond Light Source}

\begin{document}

\maketitle

\section{Caveat Lector}

These are my notes on how to run elements of the DIALS processing suite. I make no promises that anything described in here will work, or will give sensible results, or will in fact do anything. You have been warned.

\section{Introduction}

DIALS processing may be performed by either running the individual tools (spot finding, indexing, refinement, integration, exporting to MTZ) or you can run the whole lot through \verb|dials.process|, which just chains them together (and incidentally does all of the processing in P1.)

\section{dials.process}

In the simplest case, \verb|dials.process /here/are/all/images*.cbf| will do  sensible processing, with a static model of the experiment and sample, and will output a reflection file \verb|integrated.mtz| containing the intensity measurements assuming everything works correctly. Some sensible options to use are:

\begin{itemize}
\item{\verb|scan_varying=true| - allow the crystal orientation and unit cell constants to vary during the scan}
\item{\verb|--nproc=1| - only use one processor (necessary currently for data in NeXus files}
\item{\verb|intensity.algorithm=sum2d| - use 2D summation integtration rather than the default 3D summation, other algorithms are being added}
\item{\verb|shoebox.n_blocks=N| - for some N, split the data set into N blocks for integration, so as not to overload the computer}
\item{\verb|-i| - pass the images to process through the standard input e.g. from \verb|find . -name '*.cbf'| to avoid issues with limited command-line lengths}
\end{itemize}

\noindent
I have used all of these options and they appear to do sensible things. Important to make sure that what is passed in is exactly one sweep.

\section{Results}

At the start you will see:

{\small
\begin{verbatim}
********************************************************************************

                       mmmm   mmmmm    mm   m       mmmm            
                       #   "m   #      ##   #      #"   "           
                      m#mm  #   #     #  #  #      "#mmm            
                       #    #   #     #mm#  #          "#           
                       #mmm"  mm#mm  #    # #mmmmm "mmm#"           

Launching dials.process

The following tasks will be performed:
 1) Strong spots will be found (dials.find_spots)
 2) The strong spots will be indexed (dials.index)
 3) The model will be further refined (dials.refine)
 4) The reflections will be integrated (dials.integrate)
 5) The data will be exported as MTZ (dials.export_mtz)

Please be patient, this may take a few minutes

********************************************************************************

Command-line: scan_varying=true /Users/graeme/data/i04-BAG-training/th_8_2_0001

********************************************************************************
\end{verbatim}
}

\noindent
and depending on the size of your data set you may be looking at this
for a while. The first thing to happen is the peak finding, which
happens for every image in the set (this will probably be changed in
the future.)

{\small
\begin{verbatim}
********************************************************************************
Finding Strong Spots
********************************************************************************
Configuring spot finder from input parameters
--------------------------------------------------------------------------------
Finding strong spots in imageset 0
--------------------------------------------------------------------------------

Finding spots in image 0 to 540...
Extracted strong pixels from images......................................210.62s
Merged 8 pixel lists with 922661 pixels....................................0.03s
Extracted 219127 spots.....................................................1.02s
Calculated 219127 spot centroids...........................................0.98s
Calculated 219127 spot intensities.........................................0.06s
Filtered 58009 spots by number of pixels...................................0.02s
Filtered 57769 spots by peak-centroid distance.............................0.05s
\end{verbatim}
}

\noindent
The next step will be indexing, using by default 3D FFT indexing on
peaks randomly drawn from the entire set:

{\small
\begin{verbatim}
Found max_cell: 230.8 Angstrom
FFT gridding: (256,256,256)
Number of centroids used: 9104
model 1 (9100 reflections):
Crystal:
    Unit cell: (57.933, 57.953, 150.052, 90.237, 89.675, 90.287)
    Space group: P 1
    U matrix:  {{-0.3452, -0.2600,  0.9018},
                {-0.8909,  0.3929, -0.2278},
                {-0.2951, -0.8820, -0.3673}}
    B matrix:  {{ 0.0173,  0.0000,  0.0000},
                { 0.0001,  0.0173,  0.0000},
                {-0.0001,  0.0001,  0.0067}}
    A = UB:    {{-0.0061, -0.0044,  0.0060},
                {-0.0153,  0.0068, -0.0015},
                {-0.0051, -0.0152, -0.0024}}
\end{verbatim}
}

\noindent
after which some refinement will be performed - this will currently
take a little while, particularly if scan varying refinement was
specified. The main thing to look for is the RMSD values in X, Y and
Phi (in mm, mm and radians respectively) are going down and ideally
converging. Several cycles of refinement are performed with increasing
resolution limits, thus trying to achieve an increasingly accurate
result:

{\small
\begin{verbatim}
################################################################################
Starting refinement (macro-cycle 3)
################################################################################


Running refinement
------------------
0 1 2 3 4 5 6 7 8 9 10

Refinement steps
----------------
Step Nref Objective RMSD_X RMSD_Y RMSD_Phi
0 4049 1049.9 0.047277 0.03654 0.00026327 
1 4049 929.44 0.043631 0.03471 0.00027018 
2 4049 922.78 0.043567 0.034537 0.00026713 
3 4049 913.33 0.043471 0.034256 0.00026399 
4 4049 901.65 0.043308 0.033923 0.00026122 
5 4049 890.14 0.043155 0.033532 0.00025993 
6 4049 882.6 0.043147 0.033145 0.0002593 
7 4049 879.67 0.04323 0.032888 0.00025882 
8 4049 879.32 0.043279 0.032808 0.00025867 
9 4049 879.31 0.043288 0.032797 0.00025865 
10 4049 879.31 0.043288 0.032796 0.00025865 
RMSD no longer decreasing
Final refined crystal models:
model 1 (48411 reflections):
Crystal:
    Unit cell: (57.809, 57.772, 150.015, 89.981, 89.997, 90.010)
    Space group: P 1
    U matrix:  {{-0.3455, -0.2589,  0.9020},
                {-0.8914,  0.3909, -0.2293},
                {-0.2932, -0.8833, -0.3658}}
    B matrix:  {{ 0.0173,  0.0000,  0.0000},
                { 0.0000,  0.0173,  0.0000},
                {-0.0000, -0.0000,  0.0067}}
    A = UB:    {{-0.0060, -0.0045,  0.0060},
                {-0.0154,  0.0068, -0.0015},
                {-0.0051, -0.0153, -0.0024}}
\end{verbatim}
}

\noindent
After all of the refinement is complete the integration is performed -
curretly the output of integration is rather terse and includes little
in the way of diagnostics or information. In the future this will be improved:

{\small
\begin{verbatim}
********************************************************************************
Integrating Reflections
********************************************************************************
Removed invalid coordinates, 4049 remaining................................0.09s
Configurating integrator from input parameters
Integrating reflections
 Prediction type: Unknown prediction
Predicted 373210 reflections...............................................2.06s
Filtered 1847 reflections with zeta > 0.050000.............................0.00s
Calculated E.S.D Beam Divergence...........................................0.06s
Calculated E.S.D Reflecting Range..........................................0.07s
Sigma B: 0.022247
Sigma M: 0.080272
Calculated 373210 bounding boxes...........................................0.21s
Filtered 319044 reflections by detector mask...............................0.32s
Filtered 318767 reflections by zeta >= 0.05................................0.02s
Found 19 overlaps..........................................................0.08s
Extracted 318767 profiles from frames 0 -> 540...........................107.21s

Extracted 318767 profiles from block 0.....................................6.68s
Masked foreground pixels for 318767 reflections............................1.03s
Filtered 318767 reflections by detector mask...............................0.25s
Filtered 318767 reflections by zeta >= 0.05................................0.03s
Found nearest neighbours...................................................0.25s
Filtered 1757 matches by distance..........................................0.00s
Removed 0 duplicate match(es)..............................................0.00s
Calculated 318767 background values........................................6.10s
Calculated 318767 reflection centroids.....................................2.50s
Integrated 318767 reflections..............................................0.59s
Performed LP-correction on 318767 reflections.............................13.58s
\end{verbatim}
}

At the end of processing you will see something like:

{\small
\begin{verbatim}
Space group symbol from file: P1
Space group number from file: 1
Space group from matrices: P 1 (No. 1)
Point group symbol from file: 1
Number of batches: 540
Number of crystals: 1
Number of Miller indices: 318767
Resolution range: 150.015 1.169
History:
Crystal 1:
  Name: XTAL
  Project: DIALS
  Id: 1
  Unit cell: (57.8091, 57.7715, 150.015, 89.9809, 89.9967, 90.01)
  Number of datasets: 1
  Dataset 1:
    Name: FROMDIALS
    Id: 1
    Wavelength: 0.97625
    Number of columns: 11
    label        #valid  %valid     min     max type
    H            318767 100.00%  -47.00   39.00 H: index h,k,l
    K            318767 100.00%  -34.00   43.00 H: index h,k,l
    L            318767 100.00% -114.00  114.00 H: index h,k,l
    M_ISYM       318767 100.00%    1.00    1.00 Y: M/ISYM, packed partial/reject flag and symmetry number
    BATCH        318767 100.00%    2.00  538.00 B: BATCH number
    I            318767 100.00%   -8.34 3032.36 J: intensity
    SIGI         318767 100.00%    0.00   55.21 Q: standard deviation
    FRACTIONCALC 318767 100.00%    1.00    1.00 R: real
    XDET         318767 100.00%    3.38 2458.38 R: real
    YDET         318767 100.00%    2.97 2522.77 R: real
    ROT          318767 100.00%   82.09  162.45 R: real
\end{verbatim}
}

\noindent
which just shows the summary of what is in the output MTZ file. There are also indexing and refinement results which I need to add FIXME.

\section{What to do Next}

The output MTZ file \verb|integrated.mtz| may be read by pointless wich will assign the correct symmetry and so on, and may then be scaled with aimless. I have been running:

{\small
\begin{verbatim}
pointless hklin integrated.mtz hklout sorted.mtz
aimless hklin sorted.mtz hklout scaled.mtz
\end{verbatim}
}

\noindent
to get merged data for downstream analysis. The output from this will
include the merging statistics which will give some idea of the data
quality. Often passing in a sensible resolution limit to aimless is
also helpful... this should give you something like:

{\small
\begin{verbatim}
Summary data for        Project: DIALS Crystal: XTAL Dataset: FROMDIALS

                                           Overall  InnerShell  OuterShell
Low resolution limit                      149.83    149.83      1.53
High resolution limit                       1.50      8.22      1.50

Rmerge  (within I+/I-)                     0.074     0.027     0.246
Rmerge  (all I+ and I-)                    0.082     0.029     0.287
Rmeas (within I+/I-)                       0.090     0.034     0.301
Rmeas (all I+ & I-)                        0.091     0.035     0.316
Rpim (within I+/I-)                        0.051     0.020     0.171
Rpim (all I+ & I-)                         0.038     0.017     0.132
Rmerge in top intensity bin                0.028        -         - 
Total number of observations              229869      1388     11411
Total number unique                        41559       333      2036
Mean((I)/sd(I))                             17.9      37.7       6.5
Mn(I) half-set correlation CC(1/2)         0.998     0.982     0.840
Completeness                               100.0      99.7      99.9
Multiplicity                                 5.5       4.2       5.6

Anomalous completeness                      99.5     100.0      99.9
Anomalous multiplicity                       2.8       2.8       2.8
DelAnom correlation between half-sets     -0.002     0.124    -0.001
Mid-Slope of Anom Normal Probability       0.972       -         -  
\end{verbatim}
}

\end{document}
